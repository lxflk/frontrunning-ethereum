% Exposé for Bachelor's Thesis

\documentclass{scrartcl}

\usepackage[utf8]{inputenc}
\usepackage[T1]{fontenc}
\usepackage[USenglish]{babel}
\usepackage{hyperref}
\usepackage{graphicx}
\bibliographystyle{plain}

\title{Building a Dataset of Real-World Front-Running Exploits on the Ethereum Network}
\author{Luis Falke}
\date{\today}

\hypersetup{
    pdftitle={Exposé for Bachelor's Thesis},
    pdfauthor={Luis Falke, luis.falke@ruhr-uni-bochum.de},
    pdfsubject={Building a Dataset of Front-Running Exploits},
    pdfkeywords={Exposé, thesis, Ethereum, front-running, dataset},
    unicode=true,
}

\begin{document}

\maketitle

\section*{Supervision}

\begin{tabular}{ll}
    Supervisor: & Anna Piscitelli \\
\end{tabular}

\section*{Abstract}

Front-running attacks pose a significant threat to the security and fairness of decentralized finance (DeFi) applications on the Ethereum network. Despite their prevalence, there is a lack of comprehensive datasets documenting these exploits systematically. This thesis aims to address this gap by constructing a dataset of real-world front-running exploits, providing valuable insights into their mechanisms and consequences. The dataset will serve as a foundation for researchers and developers to study and mitigate front-running attacks more effectively.

\section*{Motivation}

The Ethereum blockchain has become the backbone of decentralized finance applications, where millions of dollars are exchanged daily through smart contracts. However, its transparent nature has led to a rise in front-running attacks, where adversaries manipulate the order of transactions to their advantage. These attacks exploit the time delay between transaction submission and inclusion in a block, allowing attackers to insert their own transactions and gain significant profits.

As DeFi continues to grow, front-running poses a critical challenge to both security and market fairness, particularly in decentralized exchanges and high-value smart contracts. Prior studies have demonstrated the prevalence of front-running attacks and their economic impact, with attackers gaining millions of dollars in profit~\cite{Perez21,Torres21}. Despite the growing attention, there is a lack of comprehensive datasets documenting these exploits in a systematic way, making it difficult for researchers and developers to study and mitigate the issue. This thesis aims to address this gap by constructing a dataset of real-world front-running exploits, providing valuable insights into their mechanisms and consequences.

\section*{Objectives}

The main goal of this thesis is to build a dataset that catalogs recent front-running exploits on the Ethereum network. This dataset will include:

\begin{itemize}
    \item The source code or bytecode of vulnerable smart contracts, including DeFi applications susceptible to front-running.
    \item The sequence of transactions used to exploit the vulnerability.
    \item Detailed analysis of the profit gained by the attacker, calculated based on transaction records.
\end{itemize}

The minimal goal is to collect and document at least 50 verified cases of front-running attacks. The optimal outcome involves analyzing several hundred cases and identifying common patterns or vulnerabilities exploited across different contracts. Additionally, the optimal goal includes the development of a script that can automatically detect front-running attacks.

\section*{Methodology}

To achieve these objectives, the following methodology will be employed:

\begin{enumerate}
    \item \textbf{Data Collection}: Utilize publicly available blockchain data, existing academic studies, and tools like Etherscan to identify exploit transactions.
    \item \textbf{Contract Analysis}: Use Solidity and related tools to analyze the source code or bytecode of vulnerable smart contracts.
    \item \textbf{Dataset Construction}: Organize the collected data into a structured dataset, including all relevant details of each exploit.
    \item \textbf{Pattern Identification}: Analyze the dataset to identify common patterns and vulnerabilities in front-running attacks.
    \item \textbf{Detection Script Development}: Implement a script in Python to automatically detect front-running attacks, leveraging machine learning techniques inspired by previous works such as FRAD~\cite{FRAD24} and GasTrace~\cite{GasTrace26}.
    \item \textbf{Evaluation}: Assess the effectiveness of the detection script based on accuracy and scalability.
\end{enumerate}

\section*{Expected Outcomes}

The expected outcomes of this thesis are:

\begin{itemize}
    \item A comprehensive dataset of real-world front-running exploits on the Ethereum network.
    \item Identification of common patterns and vulnerabilities exploited in front-running attacks.
    \item An automated detection script capable of identifying front-running attacks with high accuracy.
    \item Recommendations for improving security measures against front-running in smart contracts and DeFi applications.
\end{itemize}

\section*{Bibliography}

\nocite{*}
\bibliography{bibtex.bib}

\end{document}
